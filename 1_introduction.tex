\chapter{Introduction}
\label{chap:intro}

        The multiple domains are using Machine Learning for achieving the task which was previously very hard, or the result was not that efficient. There are many types of machine learning algorithms for different kinds of functions.

	    Every ML algorithms have it’s own usage and are divided into supervised learning, unsupervised learning, semi-supervised learning, and many more. The supervised learning problems are either classification or Regression. The classification is the task in which the model predicts the tag or label from the given labels for a given data point. Tagging the image and sentiment analysis are examples of the classification. In regression, the model predicts the value between an interval. Price of the house, Pollution growth-decline prediction are famous examples for regression.
	
	    Deep learning is a type of machine learning in which we use a neural network as the function approximator. The neural networks are some kind of multi-layered perceptrons(MLP). Most of the current works are using Deep learning to achieve state of art. The ImageNet challenge in which major breakthroughs were presented by AlexNet in which they won it with a margin of 15%. Since then, a lot of work has been done.
	
	    Similarly, there are works that are done on program representations and code optimization. Some classic works like program tagging, summarization, vulnerability detection, and others. On the code optimization side, device mapping in which code generated in such a way so that it can run efficiently on either of it.
	   
	    We have worked on the program representation for LLVM IR. Here, we can generate the Instruction level, function level, and program level embedding. These embedding can be used in many software engineering and optimization tasks depending on the design. We have written a program classification model in TensorFlow and PyTorch for the unseen algorithms. In the coming chapter, we will describe them in detail.	
    	
    	We have recreated the experimentation of Thread Coeresing Factor which IR2Vec and able to have improved on the state-of-the-art results. It is a compiler optimization task that depends on machine architecture. 