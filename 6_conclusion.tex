\chapter{Conclusion and Future Directions}
\label{chap:conclude}

% Conclude your thesis.

In \textbf{Chapter~\ref{chap:ch2}}, We are able to achieve better results with respect to the prior work in term of both accuracy. Additionally, the training time of the model is far better then the others. \newline

In \textbf{Chapter~\ref{chap:ch3}}, We successfully generates the vector presentation using IR2Vec for the POJ/OJ Dataset. Using this representation and the without any complex model architecture, we are able to get better results than the previous work for the supervised learning setting. On the other hand for few-shot learning, we are  able to cluster the unseen classes which were not the part of the training dataset. This work would be extended to classify or identify the binaries files. \newline

In \textbf{Chapter~\ref{chap:ch4}}, we have developed a Frontier pass in LLVM which perform loop-function inlining followed by fusion. We have done a Case study and are able show the requirement of the Frontier Pass for the Machine Learning Based Loop Distribution. \newline

In \textbf{Chapter~\ref{chap:ch5}}, we show that we have successfully developed a framework for register allocation which generates semantically correct code, which supports multiple architectures, and which can communicate between LLVM (in C++) and ML model (in Python). This framework can also be used to support the ML based optimizations in LLVM for other optimization problems, in which such kind of information to-from is required.